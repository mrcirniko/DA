\documentclass[12pt]{article}

\usepackage{listings}
\usepackage{fullpage}
\usepackage{multicol,multirow}
\usepackage{tabularx}
\usepackage{ulem}
\usepackage[utf8]{inputenc}
\usepackage[russian]{babel}

\begin{document}

\section*{Лабораторная работа №\,4 по курсу дискрeтного анализа: строковые алгоритмы}

Выполнил студент группы М80-208Б-22 МАИ \textit{Цирулев Николай}.

\subsection*{Условие}

Необходимо реализовать поиск одного образца в тексте с использованием алгоритма Z-блоков. Алфавит — строчные латинские буквы. 
\begin{itemize}
\item
    Формат ввода:
    
    На первой строке входного файла текст, на следующей — образец. Образец и текст помещаются в оперативной памяти.

\item
    Формат вывода:
    
    В выходной файл нужно вывести информацию о всех позициях текста, начиная с которых встретились вхождения образца. Выводить следует по одной позиции на строчке, нумерация позиций в тексте начинается с 0. 
\end{itemize}

\subsection*{Метод решения}

Для поиска образца в тексте необходимо было реализовать алгоритм z-блоков. Сначала строка образца и текста объединяются, и для каждой позиции вычисляется z-функция, описывающая длину наибольшего префикса строки, совпадающего с подстрокой, начинающейся в данной позиции. После этого, анализируя z-функцию строки, программа находит индексы в тексте, где образец встретился, выводя их на экран.

Описание алгоритма z-блоков: Будем идти слева направо и хранить z-блок — самую правую подстроку, равную префиксу, которую мы успели обнаружить. Будем обозначать его границы как \textit{l} и \textit{r} включительно.

Пусть мы сейчас хотим найти $z_i$, а все предыдущие уже нашли. Новый \textit{i}-й символ может лежать либо правее z-блока, либо внутри него:
\begin{itemize}
\item
    Если правее, то мы просто наивно перебором найдем $z_i$ (максимальный отрезок, начинающийся с $s_i$ и равный префиксу), и объявим его новым z-блоком.
\item
    Если \textit{i}-й элемент лежит внутри z-блока, то мы можем посмотреть на значение $z_{i-l}$ и использовать его, чтобы инициализировать $z_i$ чем-то, возможно, отличным от нуля. Если $z_{i-l}$ левее правой границы z-блока, то $z_i=z_{i-l}$ — больше $z_i$ быть не может. Если он упирается в границу, то «обрежем» его до неё и будем увеличивать на единицу.
\end{itemize}
\subsection*{Описание программы}

Исходный код программы содержится в файле main.cpp. Для хранения входных данных (образца и текста) используется контейнер стандартной библиотеки C++ \texttt{std::string}. Для хранения z-функции строки был выбран \texttt{std::vector}.

Первая строчка \texttt{main} отключает синхронизацию потоков ввода-вывода. Вторая отвязывает стандартный поток ввода от стандартного потока вывода, блягодаря чему при каждом вызове \texttt{std::cin} не сбрасывается буфер. Обе эти строчки позволяют значительно ускорить ввод-вывод в программе.

Далее программа считывает строки и объединяет их в одну. \texttt{z\_algorithm()} вычисляет z-функцию используя алгоритм z-блоков. Далее остается пройтись по полученной z-функции и вывести индексы элементов, значение которых больше или равно длине образца.

\begin{lstlisting}[language=C++]
#include <iostream>
#include <vector>

std::vector<long long> z_algorithm(std::string s) {
    long long n = (long long) s.size();
    std::vector<long long> z(n, 0);
    long long l = 0, r = 0;
    for (long long i = 1; i < n; i++) {
        if (i <= r) {
            z[i] = std::min(z[i - l], r - i + 1);
        }

        while (i + z[i] < n && s[i + z[i]] == s[z[i]]) {
            ++z[i];
        }

        if (i + z[i] - 1 > r) {
            r = i + z[i] - 1;
            l = i;
        }
    }
    return z;
}

int main() {
    std::ios_base::sync_with_stdio(false);
	std::cin.tie(0);
    std::string s;
    std::string p;
    std::cin >> s;
    std::cin >> p;
    std::vector<long long> z = z_algorithm(p + s);
    long long n = p.size();
    for (long long i = 1; i < z.size(); ++i) {
        if (z[i] >= n && i - n >= 0) {
            std::cout << i - n << '\n';
        }
    }
}
\end{lstlisting}

\subsection*{Дневник отладки}

После отправки решения в чекер была обнаружена ошибка. В ходе ручного тестирования была выяснена причина ошибки: при выводе результата не было проверки на выход индекса за пределы массива входных данных:
\begin{itemize}
\item
    Строка программы с ошибкой:
    \begin{lstlisting}[language=C++]
            if (z[i] >= n) {
    \end{lstlisting}
\item
    Исправленная строка:
    \begin{lstlisting}[language=C++]
            if (z[i] >= n && i - n >= 0) {
    \end{lstlisting}
\end{itemize}

\subsection*{Тест производительности}
В алгоритме мы делаем столько же действий, сколько раз сдвигается правая граница z-блока, поэтому асимптотика алгоритма - \textit{O(n)}.

Для лучшей наглядности приведём таблицу, в которой написанный алгоритм сравнивается с двумя встроенными алгоритмами поиска строк, доступными в стандартной библиотеке языка C++ (std::search()).
\\
\\
\begin{tabularx}{\textwidth}{ |X|X|X|X| }
    \hline
    Количество & z\_algorithm(), & std::default\_searcher, & std::boyer\_moore\_searcher, \\
    символов в тексте  & мс & мс & мс \\
    \hline
    100000 & 3240 & 1858 & 5892 \\
    1000000 & 26573 & 17369 & 58879 \\
    10000000 & 268407 & 170710 & 587061 \\
    100000000 & 2797324 & 1712765 & 5883362 \\
    \hline
\end{tabularx}
\\
\\
Как показано в таблице, время выполнения алгоритма демонстрирует практически линейную зависимость от количества символов в тексте. Это соответствует ожидаемой сложности нашего алгоритма. Однако, проведенное сравнение показывает, что алгоритмы, реализованные в std::search(), значительно превосходят по скорости написанный в данной работе алгоритм, что говорит о несовершенстве этого алгоритма, даже принимая во внимание эффективный расчёт z-функции.

Ниже приведена программа benchmark.cpp, использовавшаяся для определения времени работы функций:

\begin{lstlisting}[language=C++]

#include <iostream>
#include <random>
#include <string>
#include <algorithm>
#include <chrono>
#include <fstream>
#include <cassert>

#include "main.cpp"

int main() {
    std::random_device rd;
    std::mt19937 gen(rd());
    std::uniform_int_distribution int_dist(100, 1000);
    std::uniform_int_distribution short_dist(1, 10);
    std::uniform_int_distribution<> char_dist(0, 25);
    for (size_t k = 100000; k <= 100000000; k *= 10) {
        std::string s;
        std::string p;
        int n = int_dist(gen);
        for (size_t i = 0; i < n; ++i) {
            p += static_cast<char>('a' + char_dist(gen));
        }
        for (size_t i = 0; i < k; ++i) {
            int flag = short_dist(gen);
            if (flag == 10) {
                s += p;
                i += n;
            }
            s += static_cast<char>('a' + char_dist(gen));
        }
        std::cout << k << " & ";

        auto start1 = std::chrono::high_resolution_clock::now();
        std::vector<size_t> res1;

        std::vector<long long> z = z_algorithm(p + s);
        for (long long i = 1; i < z.size(); ++i) {
            if (z[i] >= n && i - n >= 0) {
                res1.push_back(i - n);
            }
        }
        auto finish1 = std::chrono::high_resolution_clock::now();
        auto duration1 = std::chrono::duration_cast<std::chrono::microseconds>(finish1 - start1);
        std::cout << duration1.count() << " & ";

        auto start2 = std::chrono::high_resolution_clock::now();
        std::vector<size_t> res2;
        auto it1 = s.begin();
        while (it1 != s.end()) {
            const std::default_searcher searcher(p.begin(), p.end());
            it1 = std::search(it1, s.end(), searcher);
            if (it1 != s.end()) {
                res2.push_back(it1 - s.begin());
                it1 += n;
            }
        }
        auto finish2 = std::chrono::high_resolution_clock::now();
        auto duration2 = std::chrono::duration_cast<std::chrono::microseconds>(finish2 - start2);
        std::cout << duration2.count() << " & ";

        auto start3 = std::chrono::high_resolution_clock::now();
        std::vector<size_t> res3;
        auto it2 = s.begin();
        while (it2 != s.end()) {
            const std::boyer_moore_searcher searcher(p.begin(), p.end());
            it2 = std::search(it2, s.end(), searcher);
            if (it2 != s.end()) {
                res3.push_back(it2 - s.begin());
                it2 += n;
            }
        }
        auto finish3 = std::chrono::high_resolution_clock::now();
        auto duration3 = std::chrono::duration_cast<std::chrono::microseconds>(finish3 - start3);
        std::cout << duration3.count() << " \\" << "\\";

        assert(res1 == res2);
        assert(res1 == res3);

        std::cout << '\n';

    }
    return 0;
}

\end{lstlisting}

\subsection*{Выводы}

В ходе выполнения данной работы были изучены строковые алгоритмы, в частности алгоритм z-блоков. 

\end{document}
