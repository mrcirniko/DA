\documentclass[12pt]{article}

\usepackage{listings}
\usepackage{fullpage}
\usepackage{multicol,multirow}
\usepackage{tabularx}
\usepackage{ulem}
\usepackage[utf8]{inputenc}
\usepackage[russian]{babel}
\usepackage{verbatim}

\begin{document}

\section*{Лабораторная работа №\,3 по курсу дискрeтного анализа: Исследование качества
программ}

Выполнил студент группы М80-208Б-22 МАИ \textit{Цирулев Николай}.

\subsection*{Условие}
Для реализации словаря из предыдущей лабораторной работы,
необходимо провести исследование скорости выполнения и потребления
оперативной памяти. В случае выявления ошибок или явных недочётов,
требуется их исправить.
\\\\
Результатом лабораторной работы является отчёт, состоящий из:
\begin{itemize}
\item
    Дневника выплонения работы, в котором отражено что и когда
    делалось, какие средства использовались и какие результаты были
    достигнуты на каждом шаге выполнения лабораторной работы.
\item
    Выводов о найденных недочётах.
\item
    Сравнение работы исправленной программы с предыдущей версией.
\item
    Общих выводов о выполнении лабораторной работы, полученном
    опыте.
\end{itemize}

\subsection*{Метод решения}

В ходе выполнения лабораторной работы №2 была использована утилита Valgrind. Для проверки корректности работы каждой функции и метода программы они тестировались отдельно по мере их написания, а затем — все вместе, также с использованием Valgrind.

Запуск программы с использованием Valgrind:
\begin{lstlisting}
hacker@warmachine:~/prog/da_labs/lab2/other_way/last$ g++ main.cpp
hacker@warmachine:~/prog/da_labs/lab2/other_way/last$ valgrind ./a.out –leak-check=full
==58142== Memcheck, a memory error detector
==58142== Copyright (C) 2002-2017, and GNU GPL'd, by Julian Seward et al.
==58142== Using Valgrind-3.18.1 and LibVEX; rerun with -h for copyright info
==58142== Command: ./a.out ___leak-check=full
==58142==
+ word 123
OK
+ qwerty 234567
OK
word
OK: 123
- qwerty
OK
==58142==
==58142== HEAP SUMMARY:
==58142==     in use at exit: 296 bytes in 1 blocks
==58142==   total heap usage: 5 allocs, 4 frees, 76,368 bytes allocated
==58142==
==58142== LEAK SUMMARY:
==58142==    definitely lost: 296 bytes in 1 blocks
==58142==    indirectly lost: 0 bytes in 0 blocks
==58142==      possibly lost: 0 bytes in 0 blocks
==58142==    still reachable: 0 bytes in 0 blocks
==58142==         suppressed: 0 bytes in 0 blocks
==58142== Rerun with --leak-check=full to see details of leaked memory
==58142==
==58142== For lists of detected and suppressed errors, rerun with: -s
==58142== ERROR SUMMARY: 0 errors from 0 contexts (suppressed: 0 from 0)
\end{lstlisting}
Как видно из вывода, 296 байт памяти были выделены на куче, но не освобождены. 
Воспользуемся утилитой gprof:
\begin{lstlisting}
hacker@warmachine:~/prog/da_labs/lab2/other_way/last$ g++ -pg main.cpp -o a
hacker@warmachine:~/prog/da_labs/lab2/other_way/last$ cat infile.txt | ./a > /dev/null
hacker@warmachine:~/prog/da_labs/lab2/other_way/last$ cat infile.txt | gprof ./a > profile
\end{lstlisting}
В конце программы в дереве остался один элемент, что подтверждается выводом Valgrind:
\begin{lstlisting}
==58142== total heap usage: 5 allocs, 4 frees, 76,368 bytes \end{lstlisting}

Также, как мы видим, деструктор класса TAVLTree не был вызван ни разу за все время работы программы:
\begin{lstlisting}
index % time    self  children    called     name
                                                 <spontaneous>
[1]     99.3    0.02    5.70                 main [1]
                0.00    1.88  999883/999883      TAVLTree<AVLTreeData>::remove(AVLTreeData) [3]
                0.01    1.73  999883/999883      TAVLTree<AVLTreeData>::insert(AVLTreeData) [5]
                1.12    0.01 4000001/4000001     operator>>(std::istream&, AVLTreeData&) [8]
                0.00    0.87 2000000/2000000     TAVLTree<AVLTreeData>::search(AVLTreeData) [10]
                0.00    0.03 3999766/117356317     AVLTreeData::AVLTreeData(AVLTreeData const&) [9]
                0.01    0.02 2000001/2999884     AVLTreeData::AVLTreeData() [25]
                0.02    0.00 3000000/3000000     AVLTreeData::operator[](unsigned long) [33]
                0.00    0.00 5999767/120356201     AVLTreeData::~AVLTreeData() [28]
                0.00    0.00       1/1           TAVLTree<AVLTreeData>::TAVLTree() [48]
\end{lstlisting}

Это означает, что AVL-дерево не удалялось в конце. После добавления команды \texttt{tree.destroy();} в завершение программы утечки памяти были устранены:
\begin{lstlisting}
hacker@warmachine:~/prog/da_labs/lab2/other_way/last$ g++ main.cpp
hacker@warmachine:~/prog/da_labs/lab2/other_way/last$ valgrind ./a.out –leak-check=full
==69811== Memcheck, a memory error detector
==69811== Copyright (C) 2002-2017, and GNU GPL'd, by Julian Seward et al.
==69811== Using Valgrind-3.18.1 and LibVEX; rerun with -h for copyright info
==69811== Command: ./a.out ___leak-check=full
==69811==
+ word 123
OK
+ qwerty 93485
OK
qwerty
OK: 93485

qqq
NoSuchWord
- word
OK
Word
NoSuchWord
Qwerty
OK: 93485
==69811==
==69811== HEAP SUMMARY:
==69811==     in use at exit: 0 bytes in 0 blocks
==69811==   total heap usage: 5 allocs, 5 frees, 76,368 bytes allocated
==69811==
==69811== All heap blocks were freed -- no leaks are possible
==69811==
==69811== For lists of detected and suppressed errors, rerun with: -s
==69811== ERROR SUMMARY: 0 errors from 0 contexts (suppressed: 0 from 0)
\end{lstlisting}
Во время тестирования программы была использована утилита gcov для анализа покрытия кода, что позволило оценить эффективность тестирования и выявить области, которые могут нуждаться в доработке.
Команда для компиляции программы с включенной поддержкой профилирования и генерации отчетов покрытия:
\begin{lstlisting}
hacker@warmachine:~/prog/da_labs/lab2/other_way/last/cov$ g++ -g -fprofile-arcs -ftest-coverage -o main ../main.cpp
hacker@warmachine:~/prog/da_labs/lab2/other_way/last/cov$ ./main
+ word 1
OK
+ WORLD 2
OK
word
OK: 1
wordo
NoSuchWord
- WORD
OK
word
NoSuchWord
world
OK: 2
WORLD
OK: 2
! Save /home/hacker/prog/da_labs/lab2/other_way/last/cov/file.bin
OK
! Load /home/hacker/prog/da_labs/lab2/other_way/last/cov/file.bin
OK
+ q 3
OK

+ e 5
OK
- a
NoSuchWord
- q
OK
hacker@warmachine:~/prog/da_labs/lab2/other_way/last/cov$ gcov ./main.cpp
File '../main.cpp'
Lines executed:90.00% of 210
Creating 'main.cpp.gcov'

File '/usr/include/c++/11/iostream'
No executable lines
Removing 'iostream.gcov'

File '/usr/include/c++/11/bits/stl_algobase.h'
Lines executed:97.37% of 38
Creating 'stl_algobase.h.gcov'

File '/usr/include/c++/11/bits/cpp_type_traits.h'
Lines executed:100.00% of 2
Creating 'cpp_type_traits.h.gcov'

File '/usr/include/c++/11/bits/stl_iterator_base_types.h'
Lines executed:100.00% of 2
Creating 'stl_iterator_base_types.h.gcov'

File '/usr/include/c++/11/istream'
Lines executed:100.00% of 5
Creating 'istream.gcov'

File '/usr/include/c++/11/bits/ios_base.h'
Lines executed:100.00% of 2
Creating 'ios_base.h.gcov'

File '/usr/include/c++/11/new'
Lines executed:0.00% of 1
Creating 'new.gcov'

File '/usr/include/c++/11/bits/exception.h'
Lines executed:0.00% of 1
Creating 'exception.h.gcov'

Lines executed:90.80% of 261
hacker@warmachine:~/prog/da_labs/lab2/other_way/last/cov$ lcov -t "main" -o main.info -c -d .
Capturing coverage data from .
Subroutine read_intermediate_text redefined at /usr/bin/geninfo line 2623.
Subroutine read_intermediate_json redefined at /usr/bin/geninfo line 2655.
Subroutine intermediate_text_to_info redefined at /usr/bin/geninfo line 2703.
Subroutine intermediate_json_to_info redefined at /usr/bin/geninfo line 2792.
Subroutine get_output_fd redefined at /usr/bin/geninfo line 2872.
Subroutine print_gcov_warnings redefined at /usr/bin/geninfo line 2900.
Subroutine process_intermediate redefined at /usr/bin/geninfo line 2930.
Found gcov version: 11.4.0
Using intermediate gcov format
Scanning . for .gcda files ...
Found 2 data files in .
Processing benchmark.gcda
Processing main.gcda
Finished .info-file creation
hacker@warmachine:~/prog/da_labs/lab2/other_way/last/cov$ genhtml -o report main.info
Reading data file main.info
Found 33 entries.
Found common filename prefix "/usr/include/c++"
Writing .css and .png files.
Generating output.
Processing file /home/hacker/prog/da_labs/lab2/other_way/last/main.cpp
Processing file /home/hacker/prog/da_labs/lab2/other_way/last/benchmark.cpp
Processing file 11/chrono
Processing file 11/new
Processing file 11/tuple
Processing file 11/istream
Processing file 11/bits/alloc_traits.h
Processing file 11/bits/exception.h
Processing file 11/bits/vector.tcc
Processing file 11/bits/cpp_type_traits.h
Processing file 11/bits/stl_algobase.h
Processing file 11/bits/stl_function.h
Processing file 11/bits/random.h
Processing file 11/bits/random.tcc
Processing file 11/bits/stl_tree.h
Processing file 11/bits/stl_vector.h
Processing file 11/bits/stl_iterator_base_types.h
Processing file 11/bits/stl_uninitialized.h
Processing file 11/bits/stl_iterator.h
Processing file 11/bits/stl_map.h
Processing file 11/bits/basic_string.tcc
Processing file 11/bits/move.h
Processing file 11/bits/allocator.h
Processing file 11/bits/stl_pair.h
Processing file 11/bits/uniform_int_dist.h
Processing file 11/bits/stl_construct.h
Processing file 11/bits/ios_base.h
Processing file 11/bits/stl_iterator_base_funcs.h
Processing file 11/bits/char_traits.h
Processing file 11/bits/basic_string.h
Processing file 11/ext/new_allocator.h
Processing file 11/ext/type_traits.h
Processing file 11/ext/aligned_buffer.h
Writing directory view page.
Overall coverage rate:
  lines......: 90.0% (861 of 957 lines)
  functions..: 96.1% (246 of 256 functions)
hacker@warmachine:~/prog/da_labs/lab2/other_way/last/cov$
\end{lstlisting}
После генерации файла с данными покрытия main.info с помощью lcov и создания HTML-отчета с использованием genhtml было установлено, что общее покрытие программы составляет 90\% для строк и 96,1\% для функций. Эти показатели свидетельствуют о высоком уровне тестирования и указывают на то, что основные функции программы проверены и работают корректно.

\subsection*{Выводы}

Использование Valgrind позволило оперативно обнаружить утечки памяти и гарантировать, что исправленная версия программы полностью освобождает выделенные ресурсы. Утилита gprof предоставила детализированный отчет о времени выполнения различных функций, что помогло идентифицировать узкие места в производительности. Применение gcov продемонстрировало высокий уровень покрытия кода тестами, однако также выявило участки, которые могли остаться неохваченными тестированием, что позволило составить более полные тесты.

В результате выполнения данной лабораторной работы были получены полезные навыки работы с инструментами анализа производительности и памяти. Эти навыки значительно способствуют улучшению качества программного обеспечения и повышению надежности кода. Лабораторная работа также подтвердила важность регулярного использования подобных утилит на всех этапах разработки, что позволяет своевременно обнаруживать и устранять потенциальные проблемы в коде.

\end{document}
