\documentclass[12pt]{article}

\usepackage{listings}
\usepackage{fullpage}
\usepackage{multicol,multirow}
\usepackage{tabularx}
\usepackage{ulem}
\usepackage[utf8]{inputenc}
\usepackage[russian]{babel}

\begin{document}

\section*{Лабораторная работа №\,8 по курсу дискрeтного анализа: жадные алгоритмы}

Выполнил студент группы М80-308Б-22 МАИ \textit{Цирулев Николай}.

\subsection*{Условие}
Вариант №4: Откорм бычков\\
Бычкам дают пищевые добавки, чтобы ускорить их рост. Каждая добавка содержит некоторые из N действующих веществ. Соотношения количеств веществ в добавках могут отличаться. Воздействие добавки определяется как $c_1 a_1+c_2 a_2+\cdots+c_{N a N}$, где $a_i-$ количество $i$-го вещества в добавке, $c_i$ - неизвестный коэффициент, связанный с веществом и не зависящий от добавки. Чтобы найти неизвестные коэффициенты $c_i$, Биолог может измерить воздействие любой добавки, использовав один её мешок. Известна цена мешка каждой из $M \leq N$ ) различных добавок. Нужно помочь Биологу подобрать самый дешевый наобор добавок, позволяющий найти коэффициенты $c_i$. Возможно, соотношения веществ в добавках таковы, что определить коэффициенты нельзя.

Формат ввода

В первой строке текста - целые числа $M$ и $N$; в каждой из следующих $M$ строк записаны $N$ чисел, задающих соотношение количеств веществ в ней, а за ними - цена мешка добавки. Порядок веществ во всех описаниях добавок один и тот же, все числа - неотрицательные целые не больше 50.

Формат вывода

Вывести -l если определить коэффциенты невозможно, иначе набор добавок (и их номеров по порядоку во входных данных). Если вариантов несколько, вывести какой-либо из них.

\subsection*{Метод решения}
Цель задачи — найти минимальный по стоимости набор добавок, который позволяет определить коэффициенты $c_i$.\\ Идея такова - представим входные данные в виде системы линейных алгебраических уравнений, где добавки - это $x_i$, а цена мешка добавки - это правая часть уравнения вида: $a_1_j * x_1 + a_2_j * x_2 + ... + a_i_j * x_i + ... + a_i_n * x_n = b_j$.
\\
Это значит, что для решения данной системы нужно, чтобы матрица системы имела $N$ базисных строк. Для нахождения базисных строк матрицы воспользуемся немного модифицированным методом Гаусса. Приведем матрицу к ступенчатому виду, однако всегда среди двух линейно зависимых строк будем выбирать тот, чья стоимость ниже. Данный алгоритм - жадный, так как мы каждом шаге выбираем локально оптимальное решение и в конце у нас получается оптимальное решение.

\subsection*{Описание программы}

\paragraph{Структура \texttt{TTriplet}}
\begin{verbatim}
struct TTriplet {
    std::vector<double> vec;  // Вектор веществ для добавки
    int initIdx;              // Исходный индекс добавки
    int price;                // Стоимость мешка
};
\end{verbatim}
Эта структура объединяет данные о каждой добавке:
\begin{itemize}
    \item \texttt{vec} — коэффициенты веществ в добавке.
    \item \texttt{initIdx} — индекс добавки в исходном вводе (для корректного вывода результата).
    \item \texttt{price} — стоимость добавки.
\end{itemize}

\paragraph{Ввод данных}
\begin{verbatim}
int m, n;
std::cin >> m >> n;
std::vector<TTriplet> vectors(m, TTriplet{std::vector<double>(n), 0, 0});
for (int i = 0; i < m; ++i) {
    for (int j = 0; j < n; ++j) {
        std::cin >> vectors[i].vec[j];
    }
    std::cin >> vectors[i].price;
    vectors[i].initIdx = i;
}
\end{verbatim}
\begin{itemize}
    \item Считывается \( m \)-наборов данных.
    \item Каждая добавка сохраняется в виде структуры \texttt{TTriplet}.
\end{itemize}

\paragraph{Приведение к ступенчатому виду}
\begin{verbatim}
for (int i = 0; i < n; ++i) {
    int minVecIdx = -1;
    int minPrice = std::numeric_limits<int>::max();

    for (int j = i; j < m; ++j) {
        if (vectors[j].vec[i] != 0.0 && vectors[j].price < minPrice) {
            minPrice = vectors[j].price;
            minVecIdx = j;
        }
    }

    if (minVecIdx == -1) {
        std::cout << "-1\n";
        return;
    }

    std::swap(vectors[i], vectors[minVecIdx]);
    res.push_back(vectors[i].initIdx);
\end{verbatim}
\begin{itemize}
    \item Для каждого столбца \( i \):
    \begin{itemize}
        \item Выбирается строка с минимальной стоимостью, имеющая ненулевой коэффициент в текущем столбце.
        \item Если такая строка не найдена, выводится \texttt{-1}, так как система неразрешима.
        \item Найденная строка переставляется на текущую позицию.
    \end{itemize}
\end{itemize}

\paragraph{Обнуление элементов ниже диагонали}
\begin{verbatim}
for (int j = i + 1; j < m; ++j) {
    double multiplier = vectors[j].vec[i] / vectors[i].vec[i];
    for (int k = i; k < n; ++k) {
        vectors[j].vec[k] -= vectors[i].vec[k] * multiplier;
    }
}
\end{verbatim}
\begin{itemize}
    \item Все строки ниже текущей модифицируются так, чтобы в текущем столбце \( i \) коэффициенты стали равны нулю (стандартный шаг метода Гаусса).
\end{itemize}

\paragraph{Вывод результата}
\begin{verbatim}
std::sort(res.begin(), res.end());
for (int elem : res) {
    std::cout << elem + 1 << " ";
}
std::cout << std::endl;
\end{verbatim}
\begin{itemize}
    \item Индексы выбранных строк (добавок) сортируются в порядке возрастания.
    \item Выводятся с учётом 1-индексации.
\end{itemize}

\begin{lstlisting}[language=C++]
#include <iostream>
#include <vector>
#include <limits>

using namespace std;
using ll = long long;

int main() {
    ll n;
    cin >> n;
    vector<ll> steps(n + 1, 0);
    vector<ll> c(n + 1, numeric_limits<ll>::max());
    c[1] = 0; 

    for (ll i = 2; i <= n; ++i) {
        if (c[i - 1] + i < c[i]) {
            c[i] = c[i - 1] + i;
            steps[i] = -1;
        }
        if (i % 2 == 0 && c[i / 2] + i < c[i]) {
            c[i] = c[i / 2] + i;
            steps[i] = 2;
        }
        if (i % 3 == 0 && c[i / 3] + i < c[i]) {
            c[i] = c[i / 3] + i;
            steps[i] = 3;
        }
    }
    vector<string> res;
    ll i = n;
    while (i > 1) {
        if (steps[i] == -1) {
            res.push_back("-1");
            i -= 1;
        } else if (steps[i] == 2) {
            res.push_back("/2");
            i /= 2;
        } else if (steps[i] == 3) {
            res.push_back("/3");
            i /= 3;
        }
    }
    cout << c[n] << endl;
    for (ll i = 0; i < res.size(); i++) {
        cout << res[i] << (i == res.size() - 1 ? "" : " ");
    }
    cout << endl;
    return 0;
}

\end{lstlisting}

\subsection*{Дневник отладки}

После отправки решения в чекер не было обнаружено ошибок.

\subsection*{Тест производительности}
Асимптотика написанного алгоритма - $O(MN^2)$.
Для лучшей наглядности приведём таблицу, в которой время работы алгоритма сопоставляется с вводимыми данными. Для удобства расчетов приравняем N к M.
\\
\\
\begin{tabularx}{\textwidth}{ |X|X| }
    \hline
    N & Время работы алгоритма, мс \\
    \hline
    125 & 9327 \\
    250 & 70629 \\
    500 & 587350 \\
    1000 & 4674816 \\
    \hline
\end{tabularx}
\\
\\
Как показано в таблице, время выполнения алгоритма $T = CN^3, C = const$. Это соответствует ожидаемой сложности нашего алгоритма. 
Ниже приведена программа benchmark.cpp, использовавшаяся для определения времени работы алгоритма:

\begin{lstlisting}[language=C++]

#include <iostream>
#include <vector>
#include <limits>
#include <algorithm>
#include <random>
#include <chrono>

struct TTriplet {
    std::vector<double> vec;
    int initIdx;
    int price;
};

void GaussWithPriority(int m, int n, std::vector<TTriplet> vectors) {
    std::vector<int> res;
    for (int i = 0; i < n; ++i) {
        int minVecIdx = -1;
        int minPrice = std::numeric_limits<int>::max();

        for (int j = i; j < m; ++j) {
            if (vectors[j].vec[i] != 0.0 && vectors[j].price < minPrice) {
                minPrice = vectors[j].price;
                minVecIdx = j;
            }
        }

        if (minVecIdx == -1) {
            //std::cout << "-1\n";
            return;
        }

        std::swap(vectors[i], vectors[minVecIdx]);
        res.push_back(vectors[i].initIdx);

        for (int j = i + 1; j < m; ++j) {
            double multiplier = vectors[j].vec[i] / vectors[i].vec[i];
            for (int k = i; k < n; ++k) {
                vectors[j].vec[k] -= vectors[i].vec[k] * multiplier;
            }
        }
    }

    std::sort(res.begin(), res.end());
    //for (int elem : res) {
    //    std::cout << elem + 1 << " ";
    //}
    //std::cout << std::endl;
}

int main() {
    std::random_device rd;
    std::mt19937 gen(rd());
    std::uniform_int_distribution int_dist(0, 50);
    for (int k = 125; k <= 1000; k *= 2) {
        std::cout << k << " & ";
        int m = k;
        int n = k;
        std::vector<TTriplet> vectors(m, TTriplet{std::vector<double>(n), 0, 0});
        for (int i = 0; i < m; ++i) {
            for (int j = 0; j < n; ++j) {
                vectors[i].vec[j] = int_dist(gen);
            }
            vectors[i].price = int_dist(gen);
            vectors[i].initIdx = i;
        }
        auto start1 = std::chrono::high_resolution_clock::now();
        GaussWithPriority(m, n, vectors);
        auto finish1 = std::chrono::high_resolution_clock::now();
        auto duration1 = std::chrono::duration_cast<std::chrono::microseconds>(finish1 - start1);
        std::cout << duration1.count() << " \\" << "\\ ";
        std::cout << '\n';
    }
    return 0;
}


\end{lstlisting}

\subsection*{Выводы}

В ходе выполнения данной работы была подробно изучена концепция жадных алгоритмов, а также углублены знания в области линейной алгебры, в частности, метода Гаусса. На практике полученные теоретические знания были применены для разработки эффективного алгоритма, который сочетает оптимальную временную сложность и рациональное использование памяти.\\
Особое внимание было уделено модификации классического метода Гаусса с учётом дополнительных критериев, таких как минимизация стоимости, что позволило продемонстрировать практическое применение математических методов для решения реальных оптимизационных задач. \\
Полученные результаты могут служить основой для дальнейшего изучения оптимизационных методов и применения их в более сложных задачах.

\end{document}
