\documentclass[12pt]{article}

\usepackage{listings}
\usepackage{fullpage}
\usepackage{multicol,multirow}
\usepackage{tabularx}
\usepackage{ulem}
\usepackage[utf8]{inputenc}
\usepackage[russian]{babel}
\usepackage{verbatim}

\begin{document}

\section*{Лабораторная работа №\,2 по курсу дискрeтного анализа: сбалансированные деревья}

Выполнил студент группы М80-208Б-22 МАИ \textit{Цирулев Николай}.

\subsection*{Условие}
\begin{itemize}
\item


    Необходимо создать программную библиотеку, реализующую указанную структуру данных, на основе которой разработать программу-словарь. В словаре каждому ключу, представляющему из себя регистронезависимую последовательность букв английского алфавита длиной не более 256 символов, поставлен в соответствие некоторый номер, от 0 до $2^{64} - 1$. Разным словам может быть поставлен в соответствие один и тот же номер.

    Программа должна обрабатывать строки входного файла до его окончания. Каждая строка может иметь следующий формат:
    
    + word 34 — добавить слово «word» с номером 34 в словарь. 
    
    - word — удалить слово «word» из словаря. 
    
    word — найти в словаре слово «word». 
    
    ! Save /path/to/file — сохранить словарь в бинарном компактном представлении на диск в файл, указанный парамером команды. 
    
    ! Load /path/to/file — загрузить словарь из файла. 

\item
    Вариант задания: AVL-дерево.
\end{itemize}

\subsection*{Метод решения}

Для реализации словаря был разработан класс его элементов (пар "ключ-значение"), где ключ реализован в виде статического массива символов. Для хранения и управления данными был реализован шаблонный класс \texttt{TAVLTree}, представляющий собой AVL-дерево.

AVL-дерево представляет собой бинарное дерево поиска со сбалансированной высотой: для каждого узла x высота левого и правого поддеревьев x отличается не более чем на 1.

Вставка и удаление элементов были реализованы с использованием функции \texttt{balance()}. Данная функция предназначена для балансировки AVL-дерева с использованием левого и правого поворотов. 

Балансировка высоты дерева достигается следующими двумя операциями:
\begin{itemize}
    \item
    Левый поворот дерева:
    \begin{verbatim}
        y                              x
       / \                            / \
      x   C     -------------->      A   y
     / \                                / \
    A   B                              B   C
    \end{verbatim}
    \item
    Правый поворот дерева:
    \begin{verbatim}
        x                              y
       / \                            / \
      A   y      ------------->      x   C
         / \                        / \
        B   C                      A   B
    \end{verbatim}
    где x и y — внутренние узлы, а A, B и C — AVL-деревья. 
\end{itemize}

Так как после каждого удаления и каждой вставки мы балансируем дерево, то разница высот между левым и правым поддеревом корневого узла перед каждой балансировкой <= 2. Тогда требуется рассмотреть всего 4 случая:
\begin{enumerate}
    \item
    \begin{verbatim}
        y                              x
       / \      Правый поворот        / \
      x   h     -------------->     h+1  y
     / \                                / \
   h+1 h/h+1                         h/h+1 h
    \end{verbatim}

    \item
    \begin{verbatim}
        y                       y                      z
       / \        Левый        / \      Правый        / \
      x   h      ------>      z   h     ------>     h+1  y
     / \         поворот     / \        поворот         / \
    h  h/h+1               h+1 h/h+1                 h/h+1 h
    \end{verbatim}

    \item
    \begin{verbatim}
        x                              y
       / \       Левый поворот        / \
      h   y      ------------->      x  h+1
         / \                        / \
     h/h+1 h+1                     h h/h+1
    \end{verbatim}

    \item
    \begin{verbatim}
        y                       y                      z
       / \       Правый        / \       Левый        / \
      h   x      ------>      h   z     ------>      y  h+1
         / \     поворот         / \    поворот     / \
      h/h+1 h                h/h+1 h+1             h h/h+1
    \end{verbatim}
    где x, y, z — внутренние узлы, а h и h+1 - AVL-деревья c высотами h и h+1 соответственно.
\end{enumerate}

\subsection*{Описание программы}

Исходный код программы содержится в файле main.cpp. В программе реализованы классы \texttt{AVLTreeData} и \texttt{TAVLTree}, которые обеспечивают работу словаря с использованием AVL-дерева. В классе \texttt{AVLTreeData} реализованы необходимые методы, конструкторы, деструкторы, а также перегружены операторы сравнения и индексации для корректной работы с данными:

\begin{lstlisting}[language=C++]
#include <iostream>
#include <cstdint>
#include <algorithm>
#include <cstddef>
#include <fstream>
#include <stdexcept>
#include <utility>
#include <cstring>

#define STRING_SIZE 257

class AVLTreeData {
public:
    char key[STRING_SIZE];
    uint64_t value;
    AVLTreeData() {
        std::fill_n(key, STRING_SIZE, '\0');
        value = 0;
    }
    ~AVLTreeData() {
    }
    bool operator<(const AVLTreeData& other) const {
        return strcmp(key, other.key) < 0;
    }
    bool operator==(const AVLTreeData& other) const {
        return strcmp(key, other.key) == 0;
    }
    bool operator!=(const AVLTreeData& other) const {
        return strcmp(key, other.key) != 0;
    }
    bool operator>(const AVLTreeData& other) const {
        return strcmp(key, other.key) > 0;
    }

    char& operator[](size_t idx) {
        if (idx >= STRING_SIZE) {
            throw std::out_of_range("Index out of range");
        }
        return key[idx];
    }

    const char& operator[](size_t idx) const {
        if (idx >= STRING_SIZE) {
            throw std::out_of_range("Index out of range");
        }
        return key[idx];
    }
    AVLTreeData(const AVLTreeData& other) {
        std::copy(other.key, other.key + STRING_SIZE, key);
        value = other.value;
    }

    AVLTreeData& operator=(const AVLTreeData& other) {
        if (this != &other) {
            std::copy(other.key, other.key + STRING_SIZE, key);
            value = other.value;
        }
        return *this;
    }

    friend std::ostream& operator<<(std::ostream& os, const AVLTreeData& data) {
        for (size_t i = 0; i < STRING_SIZE; ++i) {
            os << data[i];
        }
        os << ", " << data.value;
        return os;
    }
    friend std::istream& operator>>(std::istream& is, AVLTreeData& data) {
        is >> data.key;
        for (size_t i = 0; i < STRING_SIZE; i++) {
            if (data.key[i] != '\0') {
                data.key[i] = std::tolower(data.key[i]);
            }
        }
        return is;
    }
};


template <typename T> 
class TAVLTree {
public:
    class TNode {
    public:
        T data;
        int height;
        TNode* left;
        TNode* right;
        TNode(T d) {
            height = 1;
            data = d;
            left = nullptr;
            right = nullptr;
        }
        void destroy() {
            if (this != nullptr) {
                left->destroy();
                right->destroy();
                delete this;
            }
        }
    };
    TNode* root = nullptr;
    int n;
    TAVLTree() : root(nullptr), n(0) { }

    void insert(T x) { root = insertUtil(root, x); }

    void remove(T x) {
        if (root != nullptr) {
            root = removeUtil(root, x);
        }
    }

    TNode* search(const T x) { return searchUtil(root, x); }

    void printTree() { printTreeUtil(root, 0); }

    void clear() {
        if (root != nullptr) {
            destroy(root->left);
            destroy(root->right);
            delete root;
        }
        root = nullptr;
    }
    void destroy(TNode*& node) {
        if (node != nullptr) {
            destroy(node->left);
            destroy(node->right);
            delete node;
        }
    }

    void destroy() {
        destroy(root);
    }

    void LoadFromFile(std::ifstream& file) {
        clear();
        size_t fileSize;
        if (file.read(reinterpret_cast<char *>(&fileSize), sizeof(size_t))) {
            for (size_t i = 0; i < fileSize; ++i) {
                T data;
                file.read(data.key, STRING_SIZE);
                file.read(reinterpret_cast<char*>(&data.value), sizeof(data.value));
                insert(data);
            }
        }
    }

    void saveToFile(std::ofstream& file) {
        saveToFileUtil(file, root);
    }
    size_t getSize() {
        return getSizeUtil(root);
    }

private:
    size_t getSizeUtil(const TNode* node) {
        if (node != nullptr) {
            return 1 + getSizeUtil(node->left) + getSizeUtil(node->right);
        }
        return 0;
    }
    void saveToFileUtil(std::ofstream& file, TNode* node) {
        if (node == nullptr) {
            return;
        }
        file.write(node->data.key, STRING_SIZE);
        file.write(reinterpret_cast<const char*>(&node->data.value), sizeof(node->data.value));

        saveToFileUtil(file, node->left);
        saveToFileUtil(file, node->right);
    }

    void printTreeUtil(TNode* head, int space) {
        if (head == nullptr)
            return;
        space += 10;
        printTreeUtil(head->right, space);
        std::cout << '\n';
        for (int i = 10; i < space; i++)
            std::cout << " ";
        std::cout << head->data << "\n";
        printTreeUtil(head->left, space);
    }
    
    TNode* searchUtil(TNode* head, T x) {
        if (head == nullptr)
            return nullptr;
        T k = head->data;
        if (k == x) {
            return head;
        } else if (k > x) {
            return searchUtil(head->left, x);
        } if (k < x) {
            return searchUtil(head->right, x);
        }
        return nullptr;
    }

    int height(TNode* head) {
        if (head == nullptr)
            return 0;
        return head->height;
    }

    TNode* rightRotation(TNode* head) {
        TNode* newhead = head->left;
        head->left = newhead->right;
        newhead->right = head;
        head->height = 1 + std::max(height(head->left), height(head->right));
        newhead->height = 1 + std::max(height(newhead->left), height(newhead->right));
        return newhead;
    }

    TNode* leftRotation(TNode* head) {
        TNode* new_head = head->right;
        head->right = new_head->left;
        new_head->left = head;
        head->height = 1 + std::max(height(head->left), height(head->right));
        new_head->height = 1 + std::max(height(new_head->left), height(new_head->right));
        return new_head;
    }

    TNode* balance(TNode* head) {
        int bal = height(head->left) - height(head->right);
        if (bal > 1) {
            if (height(head->left) >= height(head->right)) {
                return rightRotation(head);
            } else {
                head->left = leftRotation(head->left);
                return rightRotation(head);
            }
        } else if (bal < -1) {
            if (height(head->right) >= height(head->left)) {
                return leftRotation(head);
            } else {
                head->right = rightRotation(head->right);
                return leftRotation(head);
            }
        }
        return head;
    }

    TNode* insertUtil(TNode* head, T x) {
        if (head == nullptr) {
            n += 1;
            TNode* temp = new TNode(x);
            if (temp == nullptr) {
                throw std::bad_alloc();
            }
            return temp;
        }
        if (x < head->data)
            head->left = insertUtil(head->left, x);
        else if (x > head->data)
            head->right = insertUtil(head->right, x);
        head->height = 1 + std::max(height(head->left), height(head->right));
        return balance(head);
    }
    TNode* removeUtil(TNode* head, T x) {
        if (head == nullptr)
            return nullptr;
        if (x < head->data) {
            head->left = removeUtil(head->left, x);
        } else if (x > head->data) {
            head->right = removeUtil(head->right, x);
        } else {
            TNode* r = head->right;
            if (head->right == nullptr) {
                TNode* l = head->left;
                delete (head);
                head = l;
            } else if (head->left == nullptr) {
                delete (head);
                head = r;
            } else {
                while (r->left != nullptr)
                    r = r->left;
                head->data = r->data;
                head->right = removeUtil(head->right, r->data);
            }
        }
        if (head == nullptr)
            return head;
        head->height = 1 + std::max(height(head->left), height(head->right));
        return balance(head);
    }
};



int main() {
    TAVLTree<AVLTreeData> tree;
    AVLTreeData command;
    while (std::cin >> command) {
        TAVLTree<AVLTreeData>::TNode* node;
        if (command[0] == '+') {
            AVLTreeData data;
            std::cin >> data >> data.value;
            node = tree.search(data);
            if (node == nullptr) {
                tree.insert(data);
                std::cout << "OK" << '\n';
            } else {
                std::cout << "Exist" << '\n';
            }
        } else if (command[0] == '-') {
            AVLTreeData data;
            std::cin >> data;
            node = tree.search(data);
            if (node == nullptr) {
                std::cout << "NoSuchWord" << '\n';
            } else {
                tree.remove(data);
                std::cout << "OK" << '\n';
            }
        
        } else if (command[0] == '!') {
            AVLTreeData action;
            std::cin >> action;
            AVLTreeData path;
            std::cin >> path;
            if (action[0] == 's') {   
                std::ofstream file(path.key, std::ios::binary | std::ios::trunc);
                size_t size = tree.getSize();
                file.write(reinterpret_cast<char*>(&size), sizeof(size_t));
                if (size > 0) {
                    tree.saveToFile(file);
                }
                std::cout << "OK\n";
                file.close();

            } else {
                std::ifstream file(path.key, std::ios::binary);
                tree.LoadFromFile(file);
                std::cout << "OK\n";
                file.close();
   
            }

        } else {
            node = tree.search(command);
            if (node == nullptr) {
                std::cout << "NoSuchWord" << '\n';
            } else {
                std::cout << "OK: " << node->data.value << '\n';
            }
        }
    }
    tree.destroy();
}
\end{lstlisting}

\subsection*{Дневник отладки}

При тестировании программы была обнаружена ошибка - была неверно реализована перегрузка операторов сравнения строк для класса \texttt{AVLTreeData}, при сравнении строк разных длин результат был непредсказуем, поэтому было решено использовать встроенную в язык функцию \texttt{strcmp()}.

\subsection*{Тест производительности}

Померим скорость работы кода лабораторной (будем записывать n-е количество элементов а потом удалять) и сравним с встроенной в C++ структурой \texttt{std::map}. Как видим, на больших значениях от $10^4$ до $10^6$ наш алгоритм идет наравне с библиотечным, это связано с тем, что в реализации \texttt{std::map} используется красно-черное дерево со временем операций вставки и удаления $O(log(n))$, где n - количество элементов в дереве. Такую же сложность имеют аналогичные операции в AVL-дереве.

\begin{center}
\begin{tabular}{ |c|c|c|c| }
    \hline
    Количество пар "ключ-значение" & TAVLTree, мс & std::map, мс \\
    \hline
    10000 & 139125 & 246289 \\
    100000 & 1193582 & 1654637 \\
    1000000 & 12535842 & 16596308 \\
    \hline
    \end{tabular}
\end{center}

Ниже приведена программа benchmark.cpp, использовавшаяся для определения времени работы функций:
\begin{lstlisting}[language=C++]
#include <iostream>
#include <random>
#include <string>
#include <algorithm>
#include <chrono>
#include <vector>
#include <map>

#include "main.cpp"

template <typename KeyT, typename ValueT>
struct test_pair {
    ValueT value;
    KeyT key;
    bool operator<(const test_pair& other) const {
        return key < other.key;
    }
    bool operator>(const test_pair& other) const {
        return key > other.key;
    }
    bool operator==(const test_pair& other) const {
        return key == other.key;
    }
    bool operator!=(const test_pair& other) const {
        return key != other.key;
    }
};

int main() {
    std::random_device rd;
    std::mt19937 gen(rd());
    std::uniform_int_distribution<uint8_t> key_dist;
    std::uniform_int_distribution<uint8_t> value_dist;
    std::uniform_int_distribution<> char_dist(0, 25);

    for (size_t k = 10000; k <= 10000000; k *= 10) {
        std::cout << k << " & ";
        std::map<uint8_t, uint8_t> map;
        TAVLTree<test_pair<uint8_t, uint8_t>> tree;
        test_pair<uint8_t, uint8_t> data;
        std::vector<test_pair<uint8_t, uint8_t>> vec;

        for (size_t i = 0; i < k; ++i) {
            data.value = key_dist(gen);
            data.key = value_dist(gen);
            vec.push_back(data);
        }

        auto start1 = std::chrono::high_resolution_clock::now();
        for (size_t i = 0; i < k; ++i) {
            tree.insert(vec[i]);
            
        }
        for (size_t i = 0; i < k; ++i) {
            tree.remove(vec[i]);    
        }
        auto finish1 = std::chrono::high_resolution_clock::now();
        auto duration1 = std::chrono::duration_cast<std::chrono::microseconds>(finish1 - start1);
        std::cout << duration1.count() << " & ";

        auto start2 = std::chrono::high_resolution_clock::now();
        for (size_t i = 0; i < k; ++i) {
            map[vec[i].key] = vec[i].value;
            
        }
        for (size_t i = 0; i < k; ++i) {
            map.erase(vec[i].key);  
        }
        auto finish2 = std::chrono::high_resolution_clock::now();
        auto duration2 = std::chrono::duration_cast<std::chrono::microseconds>(finish2 - start2);
        std::cout << duration2.count() << " \\" << "\\ ";

        std::cout << '\n';
        vec.clear();
        tree.destroy();
    }
}


\end{lstlisting}

\subsection*{Выводы}

В ходе выполнения данной работы были изучены методы построения сбалансированных деревьев, в частности AVL-дерева. При написании дерева наибольшую сложность имела задача написания логики вставки и удаления с балансировкой. Судя по проведенным тестам можно сказать, что AVL-дерево - достаточно быстрая и эффективная структура для хранения ключей и быстрого доступа к данным по ключу.

\end{document}
