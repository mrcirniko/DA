\documentclass[12pt]{article}

\usepackage{listings}
\usepackage{fullpage}
\usepackage{multicol,multirow}
\usepackage{tabularx}
\usepackage{ulem}
\usepackage[utf8]{inputenc}
\usepackage[russian]{babel}

\begin{document}

\section*{Лабораторная работа №\,1 по курсу дискрeтного анализа: сортировка за линейное время}

Выполнил студент группы М80-208Б-22 МАИ \textit{Цирулев Николай}.

\subsection*{Условие}

Кратко описывается задача:
\begin{enumerate}
\item
    Требуется разработать программу, осуществляющую ввод пар "ключ-значение", сортировку по возрастанию ключа указанным алгоритмом сортировки за линейное время и вывод получившейся последовательности.
\item
    Вариант задания:
        \begin{itemize}
            \item Сортировка подсчётом.
            \item Тип ключа: числа от 0 до $2^{64}-1$.
            \item Тип значения: строки фиксированной длины 64 символа, во входных данных могут встретиться строки меньшей длины, при этом строка дополняется до 64-х нулевыми символами, которые не выводятся на экран.
        \end{itemize}
\end{enumerate}

\subsection*{Метод решения}

Для хранения входных данных и блоков (карманов) был написан вектор с выделением памяти на куче. Также для сортировки элементов в карманах была написана сортировка слиянием.

\subsection*{Описание программы}

Исходный код программы содержится в файле main.cpp.
В классе \texttt{my\_vector} реализованы все необходимые для задания методы, конструкторы, деструкторы а также перегрузки необходимых для работы операторов:
\begin{lstlisting}[language=C++]
template <typename T>
class my_vector {
private:
    T * data;
    uint64_t size;
    uint64_t capacity;
public:
    my_vector() : data(nullptr), size(0), capacity(0) {}
    my_vector(uint64_t new_size) {
        data = new T[new_size];
        size = 0;
        capacity = new_size;
    }
    ~my_vector() {
        if (data != nullptr) {
            delete[] data;
            data = nullptr;
        }
    }
    uint64_t get_size() {
        return size;
    }
    my_vector(const my_vector& other) : size(other.size), capacity(other.capacity) {
        data = new T[capacity];
        std::copy(other.data, other.data + size, data);
    }

    my_vector(my_vector&& other) noexcept : data(other.data), size(other.size), capacity(other.capacity) {
        other.data = nullptr;
        other.size = 0;
        other.capacity = 0;
    }

    my_vector& operator=(const my_vector& other) {
        if (this != &other) {
            delete[] data;
            size = other.size;
            capacity = other.capacity;
            data = new T[capacity];
            std::copy(other.data, other.data + size, data);
        }
        return *this;
    }
    my_vector& operator=(my_vector&& other) noexcept {
        if (this != &other) {
            delete[] data;
            data = other.data;
            size = other.size;
            capacity = other.capacity;
            other.data = nullptr;
            other.size = 0;
            other.capacity = 0;
        }
        return *this;
    }
    void push_back(const T& elem) {
        if (size >= capacity) {
            throw std::out_of_range("Index out of range. Size: " + std::to_string(size));
        }
        data[size++] = elem;
    }
    void resize(const int k) {
        if (k <= 0) return;
        capacity += k;
        T* temp = new T[capacity];
        if (data != nullptr) {
            std::copy(data, data + size, temp);
        }
        delete[] data;
        data = temp;
    }
    T& operator[](uint64_t index) {
        return data[index];
    }
};
\end{lstlisting}
Структура \texttt{my\_pair} необходима для для хранения пар "ключ-значение". Для ключа использовался тип данных \texttt{uint64\_t}, который в большей мере подходит для заданных ограничений. Для значений был выбран массив \texttt{char} на стеке.
\begin{lstlisting}[language=C++]
struct my_pair {
    uint64_t key;
    char value[SIZE_OF_STRING];
    my_pair() {
        std::memset(value, 0, SIZE_OF_STRING);
    }
    bool operator<(const my_pair& other) const {
        return key < other.key;
    }
    bool operator<=(const my_pair& other) const {
        return key <= other.key;
    }
};
\end{lstlisting}

Функции \texttt{merge\_sort()} и \texttt{merge()} содержат реализацию алгоритма сортировки слиянием. 

\begin{lstlisting}[language=C++]
template <class T>
void merge(my_vector<T>& arr, my_vector<T>& buf, std::size_t left, std::size_t mid, std::size_t right) {
    std::size_t it1 = 0;
    std::size_t it2 = 0;
    while (left + it1 < mid && mid + it2 < right) {
        if (arr[left + it1] <= arr[mid + it2]) {
            buf[it1 + it2] = std::move(arr[left + it1]);
            it1 += 1;
        } else {
            buf[it1 + it2] = std::move(arr[mid + it2]);
            it2 += 1;
        }
    }
    while (left + it1 < mid) {
        buf[it1 + it2] = std::move(arr[left + it1]);
        it1 += 1;
    }

    while (mid + it2 < right) {
        buf[it1 + it2] = std::move(arr[mid + it2]);
        it2 += 1;
    }
    for (std::size_t i = 0; i < it1 + it2; ++i) {
        arr[left + i] = std::move(buf[i]);
    }
}

template <class T>
void merge_sort(my_vector<T>& arr, my_vector<T>& buf) {
    std::size_t count = arr.get_size();
    for (size_t i = 1; i < count; i *= 2) {
        for (size_t j = 0; j < count - i; j += 2 * i) {
            merge(arr, buf, j, j + i, min(j + 2 * i, count));
        }
    }
}
\end{lstlisting}

Функция \texttt{bucket\_sort()} содержит реализацию алгоритма сортировки. Для поддержания необходимосй скорости алгоритма для сортировки элементов карманов используется сортировка слиянием. В ходе алгоритма для экономии используемой памяти вектор \texttt{my_vector<my_pair> v} с исходными данными используется как буфер для сортировки карманов и перезаписывается.

\begin{lstlisting}[language=C++]
void bucket_sort(my_vector<my_pair> & v, uint64_t m) {
    uint64_t n = v.get_size();
    my_vector<my_pair> b[n];
    uint64_t temp1[n];
    int temp[n];
    for (uint64_t i = 0; i < n; ++i) {
        temp[i] = 0;
    }
    for (uint64_t i = 0; i < n; ++i) {
        temp1[i] = n * v[i].key / std::numeric_limits<uint64_t>::max();
        temp[temp1[i]]++;
    }
    for (uint64_t i = 0; i < n; ++i) {
        b[i].resize(temp[i]);
    }
    for (uint64_t i = 0; i < n; ++i) {
        b[temp1[i]].push_back(v[i]);
    }
    for (uint64_t i = 0; i < n; ++i) {
        if (b[i].get_size() > 1) {
            merge_sort(b[i], v);
        }
        for (uint64_t j = 0; j < b[i].get_size(); j++) {
            std::cout << b[i][j].key << "\t" << b[i][j].value << '\n';
        }
    }
}
\end{lstlisting}

Первая строчка \texttt{main} отключает синхронизацию потоков ввода-вывода. Вторая отвязывает стандартный поток ввода от стандартного потока вывода, блягодаря чему при каждом вызове \texttt{std::cin} не сбрасывается буфер. Обе эти строчки позволяют значительно ускорить ввод-вывод в программе. Далее идет ввод данных, вызывается \texttt{counting\_sort()} и выводятся результаты.

\begin{lstlisting}[language=C++]
int main() {
    std::ios::sync_with_stdio(false);
    std::cin.tie(nullptr);
    my_vector<my_pair> v(1000000);
    my_pair p;
    uint64_t m = 0;
    while (std::cin >> p.key >> p.value) {
        p.value[SIZE_OF_STRING - 1] = '\0';
        v.push_back(p);
    }
    bucket_sort(v, m);
    return 0;
}
\end{lstlisting}

\subsection*{Дневник отладки}

За время выполнения лабораторной работы было исправлено несколько проблем, из-за которых решение не проходило тесты.

В основном решение не проходило с ошибками \texttt{ML}(\texttt{Memory limit exceeded}) и \texttt{TL}(\texttt{Time limit exceeded}). 

Для решения первой проблемы пришлось использовать сортировку слиянием вместо используемой изначально сортировки вставками. Также были сведены к минимуму долгие арфиметические операции.

Чтобы решить вторую проблему, требовалось уменьшить использование памяти программой. Для этого, вместо динамического выделения памяти автоматически во время операции \texttt{push\_back}, было решено выделять память вектора заранее, просчитав точное количество требуемой памяти. Также вместо аллоцирования дополнительной памяти под буфер для сортировки карманов использовался вектор с исходными данными.

Данные исправления позволили решить проблемы, из-за которых решение не проходило чекер.

\subsection*{Тест производительности}

Померить время работы кода лабораторной и теста производительности
на разных объемах входных данных. Сравнить результаты. Проверить,
что рост времени работы при увеличении объема входных данных
согласуется с заявленной сложностью.

Карманная сортировка в лучшем случае работает за линейное время. Если посмотреть на код программы, то видно, что реализация алгоритма состоит из нескольких последовательных циклов и одного вложенного для вывода результата. Также в одном из циклов мы видим вызов функции, которая реализует сортировку слиянием элементов карманов. Из-за того, что значения ключей в массиве, который требуется отсортировать, равномерно распределены, в лучшем случае на каждый карман приходится по одному элементу, поэтому сортировать элементы в карманах в этом случае не нужно.

Для большей наглядности приведём таблицу, в которой написанная сортировка сравнивается со стандартными функция языка C++.

\begin{center}
\begin{tabular}{ |c|c|c|c| }
    \hline
    Количество пар "ключ-значение" & counting\_sort(), мс & std::sort(), мс & std::stable$\_$sort(), мс \\
    \hline
    1 & 3786 & 2572 & 6751 \\
    10 & 4151 & 5474 & 1681 \\
    100 & 496 & 6785 & 6522 \\
    1000 & 6881 & 9948 & 6867 \\
    10000 & 88688 & 125848 & 77980 \\
    100000 & 1059858 & 1563252 & 941387 \\
    \hline
    \end{tabular}
\end{center}

На больших объёмах входных данных становится заметно, что время сортировки пропорционально количеству пар "ключ-значение". Причём на достаточно больших объемах данных написанная функция оказывается заметно быстрее стандартных функций языка C++, так как те используют алгоритмы с временной сложностью O($n \log{n}$).

Ниже приведена программа benchmark.cpp, использовавшаяся для определения времени работы функций:
\begin{lstlisting}[language=C++]
#include <iostream>
#include <random>
#include <string>
#include <algorithm>
#include <chrono>
#include <fstream>

#include "main.cpp"

bool cmp(my_pair a, my_pair b) {
    return a.key < b.key;
}

int main() {
    std::ofstream out("outfile.txt", std::ios::app);
    std::random_device rd;
    std::mt19937 gen(rd());
    std::uniform_int_distribution<uint64_t> key_dist;
    std::uniform_int_distribution<> char_dist(0, 25);
    for (int k = 1; k <= 100000; k *= 10) {

        out << k << '\t';
        std::vector<my_pair> benchmark_data;
        my_vector<my_pair> arr(k);
        my_pair pair;
        for (int i = 0; i < k; ++i) {
            pair.key = key_dist(gen);
            for (int j = 0; j < SIZE_OF_STRING - 1; ++j) {
                pair.value[j] = static_cast<char>('a' + char_dist(gen));
            }
            pair.value[SIZE_OF_STRING - 1] = '\0';
            arr.push_back(pair);
            std::cout << k << '\n';
            benchmark_data.push_back(pair);
        }

        auto start1 = std::chrono::high_resolution_clock::now();
        bucket_sort(arr);
        auto finish1 = std::chrono::high_resolution_clock::now();
        auto duration1 = std::chrono::duration_cast<std::chrono::microseconds>(finish1 - start1);
        out << duration1.count() << '\t';

        auto start2 = std::chrono::high_resolution_clock::now();
        sort(benchmark_data.begin(), benchmark_data.end(), cmp);
        auto finish2 = std::chrono::high_resolution_clock::now();
        auto duration2 = std::chrono::duration_cast<std::chrono::microseconds>(finish2 - start2);

        out << duration2.count() << '\t';

        auto start3 = std::chrono::high_resolution_clock::now();
        std::stable_sort(benchmark_data.begin(), benchmark_data.end(), cmp);
        auto finish3 = std::chrono::high_resolution_clock::now();
        auto duration3 = std::chrono::duration_cast<std::chrono::microseconds>(finish3 - start3);

        out << duration3.count() << '\t';
        out << '\n';

    }
    return 0;
}



\end{lstlisting}

\subsection*{Выводы}

В ходе выполнения данной работы были изучены алгоритмы линейных сортировок, также был реализован алгоритм карманной сортировки. При написании алгоритма возникло несколько проблем, связанных со скоростью работы программы и использованием памяти, которые были успешно решены. Реализованный алгоритм сортировки имеет множество применений. Линейная временная сложность делает алгоритм эффективным для задач с ограниченным диапазоном значений и равномерно распределенными неотсортированными данными.

\end{document}
